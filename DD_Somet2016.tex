% This is "sig-alternate.tex" V2.1 April 2013
% This file should be compiled with V2.5 of "sig-alternate.cls" May 2012
%
% This example file demonstrates the use of the 'sig-alternate.cls'
% V2.5 LaTeX2e document class file. It is for those submitting
% articles to ACM Conference Proceedings WHO DO NOT WISH TO
% STRICTLY ADHERE TO THE SIGS (PUBS-BOARD-ENDORSED) STYLE.
% The 'sig-alternate.cls' file will produce a similar-looking,
% albeit, 'tighter' paper resulting in, invariably, fewer pages.
%
% ----------------------------------------------------------------------------------------------------------------
% This .tex file (and associated .cls V2.5) produces:
%       1) The Permission Statement
%       2) The Conference (location) Info information
%       3) The Copyright Line with ACM data
%       4) NO page numbers
%
% as against the acm_proc_article-sp.cls file which
% DOES NOT produce 1) thru' 3) above.
%
% Using 'sig-alternate.cls' you have control, however, from within
% the source .tex file, over both the CopyrightYear
% (defaulted to 200X) and the ACM Copyright Data
% (defaulted to X-XXXXX-XX-X/XX/XX).
% e.g.
% \CopyrightYear{2007} will cause 2007 to appear in the copyright line.
% \crdata{0-12345-67-8/90/12} will cause 0-12345-67-8/90/12 to appear in the copyright line.
%
% ---------------------------------------------------------------------------------------------------------------
% This .tex source is an example which *does* use
% the .bib file (from which the .bbl file % is produced).
% REMEMBER HOWEVER: After having produced the .bbl file,
% and prior to final submission, you *NEED* to 'insert'
% your .bbl file into your source .tex file so as to provide
% ONE 'self-contained' source file.
%
% ================= IF YOU HAVE QUESTIONS =======================
% Questions regarding the SIGS styles, SIGS policies and
% procedures, Conferences etc. should be sent to
% Adrienne Griscti (griscti@acm.org)
%
% Technical questions _only_ to
% Gerald Murray (murray@hq.acm.org)
% ===============================================================
%
% For tracking purposes - this is V2.0 - May 2012

\documentclass{sig-alternate-05-2015}


\begin{document}

% Copyright
\setcopyright{acmcopyright}
%\setcopyright{acmlicensed}
%\setcopyright{rightsretained}
%\setcopyright{usgov}
%\setcopyright{usgovmixed}
%\setcopyright{cagov}
%\setcopyright{cagovmixed}


% DOI
\doi{10.475/123_4}

% ISBN
\isbn{123-4567-24-567/08/06}

%Conference
\conferenceinfo{PLDI '13}{June 16--19, 2013, Seattle, WA, USA}

\acmPrice{\$15.00}

%
% --- Author Metadata here ---
\conferenceinfo{WOODSTOCK}{'97 El Paso, Texas USA}
%\CopyrightYear{2007} % Allows default copyright year (20XX) to be over-ridden - IF NEED BE.
%\crdata{0-12345-67-8/90/01}  % Allows default copyright data (0-89791-88-6/97/05) to be over-ridden - IF NEED BE.
% --- End of Author Metadata ---

\title{Software Engineering Artifact in Software Development Process - Linkage between bugs and issues.}
%\subtitle{[Extended Abstract]
%\titlenote{A full version of this paper is available as
%\textit{Author's Guide to Preparing ACM SIG Proceedings Using
%\LaTeX$2_\epsilon$\ and BibTeX} at
%\texttt{www.acm.org/eaddress.htm}}}
%
% You need the command \numberofauthors to handle the 'placement
% and alignment' of the authors beneath the title.
%
% For aesthetic reasons, we recommend 'three authors at a time'
% i.e. three 'name/affiliation blocks' be placed beneath the title.
%
% NOTE: You are NOT restricted in how many 'rows' of
% "name/affiliations" may appear. We just ask that you restrict
% the number of 'columns' to three.
%
% Because of the available 'opening page real-estate'
% we ask you to refrain from putting more than six authors
% (two rows with three columns) beneath the article title.
% More than six makes the first-page appear very cluttered indeed.
%
% Use the \alignauthor commands to handle the names
% and affiliations for an 'aesthetic maximum' of six authors.
% Add names, affiliations, addresses for
% the seventh etc. author(s) as the argument for the
% \additionalauthors command.
% These 'additional authors' will be output/set for you
% without further effort on your part as the last section in
% the body of your article BEFORE References or any Appendices.

\numberofauthors{1} %  in this sample file, there are a *total*
% of EIGHT authors. SIX appear on the 'first-page' (for formatting
% reasons) and the remaining two appear in the \additionalauthors section.
%
\author{
% You can go ahead and credit any number of authors here,
% e.g. one 'row of three' or two rows (consisting of one row of three
% and a second row of one, two or three).
%
% The command \alignauthor (no curly braces needed) should
% precede each author name, affiliation/snail-mail address and
% e-mail address. Additionally, tag each line of
% affiliation/address with \affaddr, and tag the
% e-mail address with \email.
%
% 1st. author
\alignauthor
Dorealda Dalipaj\\
       \affaddr{Universidad Rey Juan Carlos}\\
       \affaddr{LibreSoft Member}\\
       \affaddr{Madrid, Spain}\\
       \email{dorealda.dalipaj@urjc.es}
% 2nd. author
%\alignauthor Jesus M. Gonzalez-Barahona\\
%\titlenote{This author is the one who did all the really hard work.}\\
%       \affaddr{Universidad Rey Juan Carlos}\\
%       \affaddr{GSyC/Libresoft}\\
%       \affaddr{Madrid, Spain}\\
%       \email{jgb@gsyc.es}
%\and  
% use '\and' if you need 'another row' of author names
% 4th. author
%\alignauthor Daniel Izquierdo Cortazar\\
%       \affaddr{Universidad Rey Juan Carlos and Bitergia}\\
%       \affaddr{GSyC/Libresoft}\\
%       \affaddr{Madrid, Spain}\\
%       \email{dizquierdo@bitergia.com}
}
% There's nothing stopping you putting the seventh, eighth, etc.
% author on the opening page (as the 'third row') but we ask,
% for aesthetic reasons that you place these 'additional authors'
% in the \additional authors block, viz.
%\additionalauthors{Additional authors: John Smith (The Th{\o}rv{\"a}ld Group,
%email: {\texttt{jsmith@affiliation.org}}) and Julius P.~Kumquat
%(The Kumquat Consortium, email: {\texttt{jpkumquat@consortium.net}}).}
%\date{30 July 1999}
% Just remember to make sure that the TOTAL number of authors
% is the number that will appear on the first page PLUS the
% number that will appear in the \additionalauthors section.

\maketitle
\begin{abstract}
Researchers working with software repositories, 
often when building performance or quality models, need to recover traceability links between bug reports 
in issue tracking repositories and reviews in code review systems. 

However, too often the information stored in bug tracking repositories is not
explicitly tagged or linked to the issues reviewing them. Researchers have to
resort to heuristics to tag the data (for example, to identify if an issue is a bug report or a work item).

%Recent studies argue that software models based on imperfect datasets, with missing
%links to the code and incorrect tagging of issues, display biases
%that compromise the validity and generality of the models built on top of the datasets. 
%This risk enhances furthermore the importance of the task of recovering 
%traceability links between reports and issues. 

In this study we promote a \emph{research artifact} in software engineering, 
a reusable unit of research that can be used to support other research endeavors and has acted as a 
support material that enabled the creation of the results published in a great number of papers until now - 
linking bugs and issues 
of the code review software process. 
We present two \emph{state-of-the-practice} algorithms on how to link bugs and issues, picking as case study 
the open source cloud computing project, OpenStack. 

OpenStack enforces 
strict development guidelines and rules on the quality of the 
data in its issue tracking repository. We empirically compare the outcome of the two 
approaches, highlighting the most prominent one.
\end{abstract}

%
% End generated code
%

%
%  Use this command to print the description
%
\printccsdesc

% We no longer use \terms command
%\terms{Theory}

\keywords{Research artifact, software engineering, state-of-the-practice algorithms, open source.}

\section{Introduction}
Software process data, especially bug reports and code changes, are widely used in software engineering research. The
integration of reports and reviews, on one hand, provides valuable information on the history and evolution of a software project 
\cite{con0}. 
On the other hand, it provide the means for conducting studies that qualitatively and quantitatively analyse fundamental parameters 
for characterization of quality and performance of the code review and the most important metrics having a positive increasing 
relation with the benefits of this process.

It is used, e.g., to build models which study the impact of characteristics of issue and review discussions on the 
defect-proneness of a patch \cite{con1}, to spot defect-introducing code changes during review, to predict the number and locale 
of bugs in future software releases, or to investigate technical and non-technical 
factors influencing modern code review ( \cite{con2, con3, con4, con5, con6}).

However, to achieve such studies, 
much of the information stored in software repositories
must be processed and cleaned for further analysis. Such 
processing and cleaning is often done using assumptions 
and heuristics. A generally accepted assumption is that most issue
tracking systems contain just bug reports. Actually, 
they contain other items such as approved design specifications for 
additions and changes to the project team's code repositories or 
light weight feature specifications.

The quality of these assumptions and heuristics represents a threat to 
the validity of results derived from software 
repositories. Recent studies, by Bird et al. \cite{con7} and Antoniol et al. 
\cite{con8}, have casted doubts on the quality of data produced
using such heuristics. 
Bird et al. \cite{con7} note that when considering links between bug reports and actual 
code, there exists bias in the severity level of bugs and the experience of
developers fixing bugs, thus a quality model built using only the linked 
bugs, will be biased towards the behaviour of more experienced developers.
Antoniol et al. \cite{con8} note that many of the issues in an issue tracking system do 
not actually represent bug reports, therefore, using such data might lead to 
incorrect bug counts for the different parts of a software system.

To prevent or minimize such situations, we present two algorithms to a correct way of linking 
bugs and issues, which take into consideration the assumptions that researchers performing this task 
usually consider as true. Our approach uses a reverse engineered model of the code
review process, and extracts key information from the issue
tracking and review systems of OpenStack. The results we bring evidence,
do not involve subjective context but are material facts. As
such, they can be recorded to trace the efficiency and effec-
tiveness of the two methods.

We have used such approach to support a previous study \cite{con23} 
that brings evidence of the performance of code review by quantifying the time spent by developers to 
identify the bugs reports, the time
spent by them to carry out the reviewing process, as well as to assess the performance of code review - 
individuation of bugs.

%we would require a perfect dataset where such linkages 
%and tagging of issues was done by professional developers and with great attention to detail. 
%While such a perfect dataset might not exist, we believe that a near-ideal dataset does exist. 
%This dataset is derived from the OpenStack project. The OpenStack project follows a disciplined 
%software development process with careful attention to continuously maintaining the linkages 
%between issues and code changes through automated tool support. The strong discipline and the 
%automated tool support gives us strong confidence that this project represents a near-ideal 
%case of high quality linkage dataset that is correctly tagged.

Summarizing, this paper provides the following two contributions:
\begin{itemize}
 \item[$\bullet$] we present two fully automated approaches for linking issues from 
 bugtracking system to reviews from code review system of Open Stack, bringing out their advantages and/or 
 disadvantages,
 \item[$\bullet$] we compare the results the two methods highlighting which is the most outstanding. 
\end{itemize}

In the remainder of this paper, we first describe the necessary background notions for our work (section 2). Next, we describe the 
methodology (section 3), then present the results and discuss some conclusions (section 4). After future work (section 5), 
we conclude with aknowledgements (section 6).

\section{Background}
This section provides background information about the bug-tracking and code review environments of OpenStack and the 
tools for obtaining data from its repositories.

We choose OpenStack because it is a large project that has adopted code
reviews on a large scale, and therefore with an abundance of data coming from the engineering systems.
OpenStack uses Launchpad, an issue tracking system, which is a repository that enables
users and developers to report defects and feature requests. It allows such a reported issue to be triaged and (if deemed
important) assigned to team members, to discuss the issue
with any interested team member and to track the history of
all work on the issue. During these issue discussions, team
members can ask questions, share their opinions and help
other team members. OpenStack uses a dedicated reviewing environment, 
Gerrit, to review patches and bug fixes. It supports 
lightweight processes for reviewing code changes, i.e., 
to decide whether a developer's change is safe to
integrate into the official Version Control System (VCS). During this process, 
assigned reviewers make comments
on a code change or ask questions that can lead to a discussion
of the change and/or different revisions of the code change,
before a final decision is made about the code change. If
accepted, the most recent revision of the code change can enter
the VCS, otherwise the change is abandoned
and the developer will move on to something else.

To obtain the issue reports and code review data of these ecosystems, we used the data set 
provided by Gonz\'alez-Barahona et al. \cite{con22}. They developed the MetricsGrimoire tool to mine the repositories of
OpenStack, then store the corresponding data into a relational database. 


\section{Methodology}
Our approach uses a reverse engineered model of the code review process, and extracts 
key information from the issue tracking and code review systems.

In the specific case of OpenStack, the issues contain the identification of the bug it is reviewing. While, on the other 
hand, the comments of the bugs contain the identification of the issue that is reviewing the ticket. Consequently we 
extract these information in a new repository for further analysis.

To obtain this information we do the following steps:

\begin{enumerate}
 \item Extraction of the information from the data sources. OpenStack repositories consists in the data from the 
 issue tracking system (\emph{Launchpad}) and the data from the code review system (\emph{Gerrit}). We extracted these
 data using the \emph{Metrics Grimoire}\footnote{https://github.com/MetricsGrimoire/}, 
 a toolset for mining the repositories of OpenStack, provided and mantained by \emph{Bitergia}.
 \item Identification of data needed for the linkage. In this process we understand the infrastructure of both the issue 
 tracking system and the code review system of the project we are analyzing. We individuate the elements which carry out 
 the information that permits us to link the reports with the issues from both systems. On the behalf of the code review 
 system we found that the issues indicate the identification of the bug they are reviewing. On the behalf of the bug 
 tracking system we found that the comments of a bug reports the identification of the issue that is reviewing them. 
 Both informations allow us to link the reviews with the bugs and viceversa.
 \item Dataset preprocessing. Not all the information provided by the repositories of OpenStack is needed for this 
 analysis. First, we are interested in linking bugs with issues. Thus we can ignore all the tickets in the bug tracking system 
 that are not classified as bug. Second, our final aim is to recover the linkage for the fixed issues. Consequently we can 
 ignore all the issues for which the status is other than 'Fix Released'.   
 \item Analysis. We use Jupyter Notebook, a computational environment in which we can combine code execution, rich text, 
 mathematics, plots and rich media, along with Pandas, an open source library specialised in providing high-performance 
 data structures and data analysis. 
\end{enumerate}

\subsection{Linking issues with bugs.}
The first approach for linking recovers links from code review system to bug tracking system. 
To detect the links between issues and bugs, we first referred to the name of the branch on which a code change had 
been made, since some of them follow the naming convention "bug/989868" with "989868" being a bug identifier, which 
is called the \emph{related artifact}. 
Applying regular expressions we extracted the bug identification, and then proceded with the linkage. Summarizing 
the \emph{state-of-practice} algorithm for linking from the code reveiw system to the bug tracking system is:

\begin{list}{}{}

\item[$\space$] identify the \emph{related artifact};

apply regular expresions to extract \emph{bug\_id} from \emph{related artifacts};

link between issues and bugs matching where bug identification is \emph{bug\_id}.

\end{list}

The most inconvenient drawback in this approach is that, even though OpenStack project follows a disciplined 
software development process with careful attention to continuously reporting the information that permits 
traceability between issues and bugs, in practice, often, developers do not report such related artifacts, 
resulting in missing links.

After applying the above approach to OpenStack, we identified 72\% of the links with issues to the bugs they 
review.

\subsection{Linking bugs with issues.}
The second approach for linking recovers links from bug tracking system to code review system.

The first problem that arises in this case is how to identify the tickets classified as bugs. 
In Launchpad, besides bugs reports, the developers can work with \textit{specifications} 
(approved design specifications for additions and changes to the project code repositories) 
and \textit{blueprints} (lightweight feature specifications).
Identifying which of the reports have been classified as describing bugs is not a trivial task.
Tickets usually are commented. Reviewers do discuss about having found a bug in a certain report or not. 
But, analysing the comments of a ticket is not the most efficient way for extracting its classification, 
not only because we will not identify 100\% of the tickets but we risk false positives too.

By manual analysis of the tickets and studying the Launchpad work flow, we came across a pattern 
in the evolution of a report states, with regards to confirming new bugs:

\begin{list}{}{}

\item[$\space$] a) when a ticket, stating a possible bug, is opened in Launchpad, its status is set to 
\textit{New};
\item[$\space$] b) if the problem described in the ticket is reproduced, the bug is confirmed as genuine and the 
ticket status changes from \textit{New} to \textit{Confirmed};
\item[$\space$] c) only when a bug is confirmed, the status then changes from \textit{Confirmed} to \textit{In Progress} 
the moment when an issue is opened for review in Gerrit.
  
\end{list}

Thus, we analysed the Launchpad repository searching for tickets that match with this pattern. 
These are the tickets that have been classified as bug reports. Once identified, we extracted them in a 
new repository for further analysis. 

Next step, we linked tickets to reviews using the information that we find in the comments of the tickets. 
Whenever a review receives a proposal for a fix, or a merge for a fix, it is reported in the 
comments of the respective ticket. 
Precisely, a merge comment looks like the following: 
\\

Reviewed: https://review.openstack.org/13083

Committed: https://git.openstack.org/cgit/openstack

/nova/commit/?id=be58dd8432a8d12484f5553d79a02

e720e2c0435

Submitter: Jenkins

Branch: master ...
\\

We can see that in the first line, clearly, we are provided with the link to the issue in Gerrit. 

A problem that arises in analysing the comments is that, for some ticket, they are a summary of some commit history. 
Therefore, in these cases, we find more than a match with the pattern we are looking for within the body of 
the comment, while the commit itself is not a merge in the master branch of the project that originated the ticket, 
consequently not the correct result. 

However there is a fixed format of the comments that report a merge (which is the one you can see in the example above).
In this format, the information related to the review is stated at the very beginning of the 
comment. Manually analysing the tickets in Launchpad, we have seen that they are found in the first 6 rows of the comment.

Thus the first step is trunking the comments, so that we extract just the first 6 lines from every one of them. 
Then, we are sure we will identify the right review. 

Summarizing the \emph{state-of-practice} algorithm for linking from the the bug tracking system to 
the code reveiw system is:

\begin{list}{}{}

\item[$\space$] identify \emph{bug reports};

exctract \emph{issue\_id}, the information related to the issue that is reviewing the bug from the comments of the bug;

link between bugs and issues where issue identification is \emph{issue\_id}.

\end{list}

After applying the above approach to OpenStack, we identified 90.2\% of the links with bugs to the issue reviewing them. 

Furthermore, analysing the missing merges from the resulting dataset by picking up arbitrarily tickets, we found that 
the reason they could not be linked is 
that some of the gerrit identification, of the issue reviewing the bug, are missing from the 
Gerrit repository.

For example: 
\\

the bug:

https://review.openstack.org/\#/q/topic:bug/

1253497 
\\

is fixed in the issue:

https://review.openstack.org/

\#/c/92547/ ...
\\

but the issue id 92547 is not found in the Gerrit database.
\\

However, we would note that, thanks to the heuristics applied, there are no false positives in the 
resulting dataset in this case. 

\subsection{Comparison of Results and Conclusions}
In this paper we present two \emph{state-of-the-practice} algorithms regarding what is by now 
a \emph{research artifacts} in software engineering. 
We discussed it's reusability and usefulness to support other research endeavors in section 1. 
We run the two algorithms on a large open software project, such OpenStack and obtained two distinct results 
(Table \ref{tab:2}), showing 
clearly that one approach outstands the other.
We hope these results will be a guideline to other researchers in future helping them to decide how to 
recover the links at the best of their advantage. While, on one hand, we tested our approach on a single case study, 
OpenStack, on the other what enfroces our confidence that this case acts as a representative for the category is 
that code review process follows the same principles and practices everywhere. 

\begin{table*}
\centering
\caption{The results of the two state-of-the practice algorithms applied on OpenStack (OS) history from July 2010 - January 2016.}
\begin{tabular}{|c|c|l|l|} \hline
Number of tickets in OS&Nr. of tickets fixed in OS&1st Approach (Gerrit-Launchpad)&2nd Approach (Launchpad-Gerrit)\\ \hline
\texttt{88421} & 70587& 72\%& 90.2\%\\ \hline

\end{tabular}
\label{tab:2}
\end{table*}
% end the environment with {table*}, NOTE not {table}!

\section{Future Work}
We would like to continue to explore the power of the information from the code review process. 
In section 1 we mentioned a study that noted that when considering links between bug reports and actual 
code, there exists bias in the severity level of bugs and the experience of
developers fixing bugs, thus a quality model built using only the linked 
bugs, will be biased towards the behaviour of more experienced developers.

To prevent and minimize such situations, we would require a perfect dataset where such linkages 
and tagging of issues was done by professional developers and with great attention to detail. 
While such a perfect dataset might not exist, we believe that a near-ideal dataset does exist. 
This dataset is derived from the OpenStack project. 

The OpenStack project follows a disciplined 
software development process with careful attention to continuously maintaining the linkages 
between issues and code changes through automated tool support. The strong discipline and the 
automated tool support gives us strong confidence that this project represents a near-ideal 
case of high quality linkage dataset that is correctly tagged.

Therefore, in our immediate future work we aim to verify the
effects of such biases for the OpenStack project, to see if even with a near-ideal dataset, biases
do exist - leading us into verifying the conjecture if biases are more likely
a symptom of the underlying software development process 
instead of being due to the used heuristics.


%ACKNOWLEDGMENTS are optional
\section{Acknowledgments}
We would like to thank the SENECA EID project, which is funding this research under Marie-Skodowska Curie Actions.

% the syntax of "referenc.tex" for your own citations
\input{referenc}

\end{document}
